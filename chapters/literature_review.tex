\chapter{Literature Review}

This chapter exposes some of the works whose content is relevant to
the context of this dissertation. Section
\ref{sec:privacy_pervasive_computing} begins by explaining the concept
of privacy in pervasive computing and it's importance. It then follows
with the presentation of some of the approaches already used in the
field of pervasive computing regarding privacy. Lastly, Section
\ref{sec:algorithms_and_data_structures} provides some background to
the algorithms and classic data structures used in our approach to ensure
privacy.

\section{Privacy in Pervasive Computing}
\label{sec:privacy_pervasive_computing}

\subsection{Definition of Privacy}
\label{sec:definition_privacy}

As location based services become more popular and compelling, there
is an increasing temptation to give away location data. However, along
with this temptation comes an increasing concern about location privacy.

Beresford and Stajano ~\cite{1186725} define location privacy as:
\begin{quotation}
  the ability to prevent other parties to from learning one's
  current or past location.
\end{quotation}
Their definition implies that a person whose location is being
monitored must be able to control who has access that information.
It also acknowledges that both past and present location information
are important. While current location information might enable an
attacker to find out where a person is, past data can allow him/her to
discover who the person is and  where does she live/work among other
things.

According Duckham and Kulik~\cite{duckham2006location} location
privacy is:
\begin{quotation}
  a special type of information privacy which concerns the claim
  of individuals to determine for themselves when, how, and to what
  extent location information about them is communicated to others.
\end{quotation}
Their definition is based upon Westin's~\cite{westin1968privacy}
definition of information privacy:
\begin{quotation}
  the claim of individuals, groups or institutions to determine for
  themselves when, how and to what extent information about them is
  communicated to others.
\end{quotation}
Duckham and Kulik's definition captures several aspects about
location information and the way it can be shared:
\begin{itemize}
\item When: A user might have different concerns regarding her present
  and past locations. For instance, the user may not care as much
  about revealing her past locations as it does about its present
  location.
\item How: A user might be comfortable with manual location requests
  from her friends, however she may not want to have her friends
  alerted automatically when she enters a casino or a bar.  
\item Extent: A user might prefer to have her location reported as an
  ambiguous region rather than as a precise point.
\end{itemize}
These different characteristics are the topic of many different
computational schemes for protecting the privacy of the users.
Examples of such schemes are the use of pseudonyms instead of actual
names, the deliberate addition of noise to the location data and the use
of regions for location reporting instead of specific points.

\subsection{Related Work}
\label{sec:related_work}

% In the context of ubiquitous computing, there are already several
% techniques aimed at ensuring the privacy of data regarding the
% location of the users.

Nowadays mobile phones come equipped with lots of sensors
(accelerometer , gyroscope, thermometer, GPS) as well as communication
interfaces (Bluetooth, WiFi). The growing number of mobile phones
together with their mentioned capabilities makes them viable building
blocks for urban infrastructures~\cite{kostakos2009understanding} as
well as targets for services such as Location Based Services(LBS).
targets for 
well as 

as well as the 
These capabilities along with the
growing number of mobile phones make
Due to the growing number of web enabled, context and location aware mobile
devices, there is nowadays an increasing number of services and infrastructures built around these
capabilities has been increasing substantially. Among the offered
services, one of the most popular 


there has been an increase in the popularity and number of
Location Based Services (LBS)~\cite{wang2008location}. However as
location tracking c
Along with these 

As mobile phones have matured as a computing platform and acquired
richer functionality, these advancements often have been paired with
the introduction of new sensors. For example, accelerometers have
become common after being initially introduced to enhance the user
interface and use of the camera. They are used to automatically
determine the orientation in which the user is holding the phone and
use that information to automatically re-orient the display between a
landscape and portrait view or correctly orient captured photos during
viewing on the phone.

Figure 1 shows the suite of sensors found in
the Apple iPhone 4. The phone’s sensors include a gyroscope, compass,
accelerometer, proximity sensor, and ambient light sensor, as well as
other more conventional devices that can be used to sense such as
front and back facing cameras, a microphone, GPS and WiFi, and
Bluetooth radios. Many of the newer sensors are added to support the
user interface (e.g., the accelerometer) or augment location-based
services (e.g., the digital compass). The proximity and light sensors
allow the phone to perform simple forms of context recognition
associated with the user interface. 

The proximity sensor detects, for
example, when the user holds the phone to her face to speak. In this
case the touchscreen and keys are disabled, preventing them from
accidentally being pressed as well as saving power because the screen
is turned off. Light sensors are used to adjust the brightness of the
screen. The GPS, which allows the phone to localize itself, enables
new location-based applications such as local search, mobile social
networks, and navigation. The compass and gyroscope represent an
extension of location, providing the phone with increased awareness of
its position in relation to the physi- cal world (e.g., its direction
and orientation) enhancing location-based applications.

Not only are these sensors useful in driving the user interface and
providing location-based services; they also represent a significant
opportunity to gather data about people and their environments. For
example, accelerometer data is capable of characterizing the physical
move- ments of the user carrying the phone

 The progress registered


\section{Algorithms and data structures}
\label{sec:algorithms_and_data_structures}
This section 
\subsection{Hash Sketches}
\label{sec:hash_sketches}

Hash Sketches are simple probabilistic data structure with which we 
obtain the cardinality of multisets. They allow for 
%In our case this means we can count the
%number of different people that crossed a gate.
Much like Bloom Filters there are several variants of this algorithm.
Despite being different, all of these variants have at least one bit array
and use some kind of hash function to map elements to positions in the
aforementioned array(s).
vc sys
Hash Sketches have a small memory footprint, the ability to estimate the
cardinality in a single pass over the set, as well as $O(1)$ complexity to
add a new element and $O(m)$ complexity to estimate the cardinality (where
$m$ is the size of the bit array).

In the Gate Counter Scenario, we tested several versions of sketches: LogLog
Sketches \cite{Durand:2003tc}, HyperLogLog Sketches \cite{Fusy:2007um}, Linear
Counting Sketches \cite{Whang:1990uh}, Robust In Network Aggregation Linear
Counting Sketches (RIA-LC) \cite{Fan:2008wl,YaoChungFanArbeeLPChen:2010to} and
Robust In Network Aggregation Dynamic Counting Sketches (RIA-DC)
\cite{YaoChungFanArbeeLPChen:2010to}.

LogLog Sketches are similar to the Probabilistic Counting algorithm
presented in \cite{Flajolet:1985wd} since both use several small bit arrays
(called buckets) instead of a single bit array. The main difference is that
LogLog Sketches are much less memory consuming at the expense of some
accuracy. Their name derives from the fact that each small bit array has
size close to $log(log(N))$, being $N$ the number of distinct elements. The
estimate of the cardinality is obtained using the average of the several
small bit arrays.

HyperLogLog Sketches are an improvement over LogLogSketches. Using the same
number of bits as LogLog Sketches, HyperLogLog Sketches are able to provide more
accurate results. According to the authors in \cite{Fusy:2007um}, this
improvement is accomplished by using \emph{harmonic means} instead of
\emph{geometric means} in the evaluation function.

Linear Counting Sketches use only a single bit array. Their name comes from the
fact that they have $O(N)$ size, meaning their size grows linearly with the
number of distinct elements $N$. When compared to LogLog Sketches, in Linear
Counting Sketches the size is a drawback, but they work better for small
cardinality sets.

Both RIA-DC and RIA-LC sketches are based on the Linear Counting
Sketches. While RIA-LC can be seen as a slightly improved/simplified version of
Linear Counting Sketches, RIA-DC Sketches have the unique ability to
merge sketches of different sizes. However, this ability comes with a
price. RIA-DC Sketches assume that there is no overlap of elements belonging to
different sketches, meaning that if we merge two different RIA-DC sketches with
elements in common, those elements will be counted twice in the final aggregate.

\subsection{Bloom Filters}
\label{sec:bloom_filters}


\subsection{Vector Clocks}
\label{sec:vector_clocks}


%%% Local Variables: 
%%% mode: latex
%%% TeX-master: "../thesis"
%%% End: 
