\chapter{Conclusion}
\label{cha:conclusion}

In this dissertation, we presented two Bluetooth tracking scenarios
where the use of stochastic summarizing techniques ensures the privacy
of the users.

In the first scenario, we explored the use of both Hash Sketches and
Bloom Filters as privacy preserving solutions for gate counting
scenarios.  It was shown that, as a result of their ability to ignore
repeated sightings and to merge different counters, these techniques
are able to provide accurate values for the number of unique sightings
across one or several gate counters.  Moreover, we analyzed the
performance of each technique regarding their size, accuracy and
aggregation capabilities and had a little discussion about some gate
counting scenarios where each technique might be a good
fit. Altogether, these techniques proved to be a good solution for gate
counting scenarios, preserving the privacy of the users (no need to
store unique identifiers). All while being much more efficient (one
order of magnitude in some cases) in terms of space than the naive
approach (plain storage of MAC addresses).


For the second scenario, we introduced a new technique, Precedence
Filters. This technique allowed us to gather accurate information
about the visiting patterns of groups of users all while ensuring that
the information kept about individuals has a level of uncertainty
compatible with plausible deniability.  This affirmation is confirmed
by the results obtained from our tests as they show that depending on
the number of users and places visited (trace size) it is possible to
have individual information uncertainty of around $50\%$ while
maintaining a good enough accuracy $  <15\%$ for aggregate information
regarding the visiting patterns all users as a whole.


Even though both our scenarios are Bluetooth based, it must be noted
that the algorithms and data structures presented in this dissertation
can also be applied in other scenarios with different tracking
technologies. These Bluetooth scenarios can be seen simply as ``proofs of
concept'', which demonstrate that stochastic summarizing techniques are
suitable for large scale sensing scenarios where privacy is paramount
and must be preserved.


\subsubsection{Future Work}
\label{sec:future-work}

Concerning Precedence Filters, there are two main issues that should be
addressed.

The first is related to the assumption stating there are
no communication failures in the system. This is a strong a requirement,
hard to fulfill specially in large scale distributed scenarios.  In
order to loosen our requirements, we could implement a new benchmark which is
able to test the occurrence of network partitions.  This would allow
us to test how Precedence Filters cope with such scenarios. Given that
PFs are based upon Vector Clocks, it should be possible to leverage
their capabilities to detect the occurrence of conflicts (deriving
from the partition), and then the simplest solution would be to discard
all the related information.

The other issue pertains the data sets. All the test we
ran were based upon a single data set, so it would be useful to test
PFs using other data sets taken from real world scenarios. Also, it would
helpful to find a statistical distribution that fits the data set we
already have, thus allowing the synthetic creation of scenarios with
an arbitrary number of scanning nodes.


%%% Local Variables:
%%% mode: latex
%%% TeX-master: "../thesis"
%%% End:
