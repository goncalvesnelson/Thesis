\chapter{Introduction}
\label{cha:introduction}

Propelled by the increasing usage of web enabled, context and location
aware mobile devices, context aware services specially location based
services (LBS) are becoming remarkably popular. The developments seen
in technologies like GPS, GSM, Wifi and Bluetooth also worked as a
catalyst in this process. Using these technologies it is possible to
obtain information about the location of users. This type of
information can be used both by service providers and users. Service
providers can process this information to attain a better
understanding of the users behavior (e.g. by analyzing their movement)
and improve their current services as well as plans for future
services and infrastructures. Users, on the other hand, can use
location data to get personalized information according to their
location. However, the continuous monitoring, processing and storage
of location data can create privacy related problems. Location
information can expose sensitive information about users as it
constitutes a \emph{quasi-identifier}\footnote{Data that by itself
  does not uniquely identify a user but that in association with other
  information might} according to Bettini et
al.~\cite{bettini2005protecting}. 

Examples of attacks on privacy using
recorded location data are not hard to find. Using a week-long
database with GPS traces of Detroit drivers whose names were replaced
with pseudonyms, Hoh et al.~\cite{hoh2006enhancing} were able to
determine the location of 85\% of the drivers' homes. Hightower et
al.'s BeaconPrint algorithm~\cite{hightower2005learning} is able to
accurately recognize the location of a user using only Wifi and GSM
information. Researchers have also shown that looking at the historic
of someone's location data is possible to predict their future
whereabouts~\cite{krumm2006predestination,froehlich2008route}.

% In this dissertation we talk about two scenarios based upon Bluetooth
% tracking. The reasons for that was chosen as the tracking technology 
\todo{Melhorar ligação entre o uso de cenários Bluetooth e o contexto.
Onde falar das motivações que nos levaram a escolher Bluetooth?}

\section{Context}
\label{sec:context}

As Bluetooth becomes more and more pervasive, there is a growing
potential to leverage on the possibilities offered by Bluetooth
scanning as a flexible infrastructure for situated interaction and a
general purpose platform for massive sensing and actuation in physical
spaces.

Bluetooth sensing is based on a discovery process through
which a device can inquire about the presence of other nearby devices.
If those devices are in \emph{discoverable} mode, they will respond
with their Bluetooth addresses (48 bit MAC), and possibly additional
information, such as the device name, the device type (e.g. cellphone
or computer) and available services.

A Bluetooth scanner is a device that periodically scans nearby
Bluetooth devices. Each time a device is detected, it registers and
timestamps that event. Later, this information can be used by external
services or applications. Multiple Bluetooth scanners spread all over
a physical space could thus serve as collection points for Bluetooth
sightings, providing a major tool for observing, recording, modeling
and analyzing that space, physically, digitally and socially
\cite{Oneill:2006vq}.

An infrastructure of this nature can be built from the scratch given
the increasingly smaller cost of the technological equipment required.
Alternatively, it can be built upon the large number of Bluetooth
scanners already present in some urban environments. These scanners
are owned by many entities and they serve very diverse purposes, such
as proximity marketing, device localization or OBEX-based interaction.
They could be used, without any additional cost, as nodes in a large
scale collaborative sensing infrastructure. Each node would still scan
for its own purposes, but it would also share part of the generated
data with a central service that would then be able to produce
aggregate information of mutual interest. Both strategies make the
creation of this type of infrastructure relatively simple and feasible
in the short term.

The challenge, however, is how to enable this type of large scale
sensing without creating an overwhelming privacy threat to the users.
This dissertation aims to find if stochastic summarizing techniques
are suitable for this challenge. To assess their viability, we used
them in two different Bluetooth scanning scenarios.

% There are already several techniques whose purpose is to ensure
% location privacy in the exchange of information between users and
% service providers. This dissertation aims to find if stochastic
% summarizing techniques are suitable as building blocks for privacy
% preserving systems. To assess their viability, we used them in two
% different Bluetooth scanning scenarios. 

\section{Structure and Contributions}
\label{sec:structure}

\subsubsection{Structure}
\label{sec:structure} This dissertation is organized as follows:
Chapter \ref{chap:literature_review} presents some of the related work
in the field of location privacy followed by a brief description of
some algorithms/data structures used in our privacy preserving
approaches. Chapters \ref{cha:gate-counting} and
\ref{cha:causality-tracking} describe both our Bluetooth tracking
scenarios, Collaborative Gate Counting and Causality Tracking, their
requirements and the effectiveness of several stochastic summarizing
techniques in dealing with them. Chapter \ref{cha:conclusion} draws
some general conclusions about our work and presents some future
research directions to be followed.

\subsubsection{Contributions}
\label{sec:contributions}
As a result of the work developed in the context of this
dissertation, two full papers were written. The first one
\cite{gonccalves2011privacy} was published in the OTM 2001 Workshop.
The second paper \cite{inforum2012} was published in INFORUM 2012
where it won the prize of best paper in the Mobile and Ubiquitous
Computing session.

%%% Local Variables: 
%%% mode: latex
%%% TeX-master: "../thesis"
%%% End: 
