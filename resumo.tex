\chapter*{Resumo}
Motivados pelo cada vez maior número de dispositivos móveis, os
serviços baseados na localização estão a tornar-se cada vez mais
populares.  Estes serviços utilizam informação acerca da
localização física dos utilizadores, normalmente para fins comerciais
ou informativos. Contudo, e particularmente para cenários de larga
escala, este tipo de serviços pode constituir um risco para a
privacidade dos utilizadores. Informação relacionada com localização
dos utilizadores pode ser utilizada de forma direta ou indireta
(associada com outra informação) para revelar informação privada
acerca dos utilizadores, podendo até ser suficiente para revelar a
identidade dos mesmos.  Este facto pode levar à rejeição deste tipo de tecnologias.

Existem contudo, maneiras não triviais de guardar informação sem
comprometer a privacidade dos utilizadores.  Nesta dissertação, apresentamos
dois cenários Bluetooth, onde o problema da privacidade é
solucionado através do uso de técnicas de sumarização estocásticas.

No primeiro cenário, \textit{Gate Counting}, o objetivo é obter
contagens precisas para o número de avistamentos de dispositivos distintos,
tentando em simultâneo reduzir a quantidade de informação
recolhida. Para esse efeito, fazemos uma análise a várias
técnicas de contagem estocásticas que não só fornecem contagens
precisas para o número de dispositivos únicos, como também
garantias de privacidade, tudo de uma forma eficiente em termos de
espaço.

Para o segundo cenário, \textit{Causality Tracking}, o objetivo é
estudar os padrões de mobilidade humanos, ao mesmo tempo que, também
se tenta minimizar a quantidade de informação recolhida.  Com este
propósito, desenvolvemos os Filtros de Precedência, uma nova técnica
capaz de fornecer resultados precisos sobre a popularidade de
determinados percursos/caminhos específicos, sem comprometer a
privacidade individual dos utilizadores.

Com base nestes cenários, esta dissertação demonstra que as técnicas
de sumarização estocásticas são meios viáveis para a anonimização de
informação baseada na localização.

%%% Local Variables:
%%% mode: latex
%%% TeX-master: "thesis"
%%% End:
