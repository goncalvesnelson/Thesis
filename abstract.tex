\chapter*{Abstract}

Driven by the pervasiveness of mobile devices, location based services
are becoming increasingly popular. These services use
information about the physical location of users, usually with
commercial or informative purposes.  However, and particularly for
large scale scenarios, this type of services may pose a risk the
privacy of the users. By using location information either
directly or indirectly (associated with other information), it is
possible to expose personal information that users wish to
keep private or even to uncover their identities. This may lead to
the rejection of these types of technologies.

There are however, non trivial ways to store information without
compromising the users' privacy.  This dissertation presents two
Bluetooth scenarios where stochastic summarizing techniques are used
as a solution to the privacy problem.

In the first scenario, Gate Counting, the goal is to provide accurate
counting for the number of unique devices sighted while trying to
minimize the amount of collected information. For that purpose,
we provide an analysis of several stochastic counting techniques that
not only provide an accurate count for the number of unique devices,
but offer privacy guarantees as well, all in a space efficient way.

For the second scenario, Causality Tracking, the objective is to study
human mobility patterns, also while minimizing the quantity of data
gathered.  For this purpose, we developed Precedence Filters, a new
technique, which is able to provide accurate results regarding the
popularity of specific routes without compromising the individual
privacy of the users.

Based on these scenarios, this dissertation demonstrates that
stochastic summarizing techniques are viable means to the
anonymization of location information.


%%% Local Variables:
%%% mode: latex
%%% TeX-master: "thesis"
%%% End:
